\documentclass[10pt,twocolumn,letterpaper]{article}


%-------------------------------------------------------------------------
%%%%%%%%% PACKAGES
%-------------------------------------------------------------------------
\usepackage{latex/cvpr}
\usepackage{times}
\usepackage{epsfig}
\usepackage{graphicx}
\usepackage{amsmath}
\usepackage{amssymb}

% Include other packages here, before hyperref.

% If you comment hyperref and then uncomment it, you should delete
% egpaper.aux before re-running latex.  (Or just hit 'q' on the first latex
% run, let it finish, and you should be clear).
\usepackage[breaklinks=true,bookmarks=false]{hyperref}



%-------------------------------------------------------------------------
%%%%%%%%% CVPR
%-------------------------------------------------------------------------
\cvprfinalcopy % *** Uncomment this line for the final submission

\def\cvprPaperID{****} % *** Enter the CVPR Paper ID here
\def\httilde{\mbox{\tt\raisebox{-.5ex}{\symbol{126}}}}

% Pages are numbered in submission mode, and unnumbered in camera-ready
%\ifcvprfinal\pagestyle{empty}\fi
%\setcounter{page}{4321}



%------------------------------------------------------------------------%
%%%%%%%%%-------------------BEGIN DOCUMENT-----------------------%%%%%%%%%
%------------------------------------------------------------------------%
\begin{document}


%-------------------------------------------------------------------------
%%%%%%%%% TITLE
%-------------------------------------------------------------------------
\title{Project Milestone}

\author{
  Caleb Geniesse\\
  Stanford University\\
  Stanford CA 94305\\
  {\tt\small calebgeniesse@stanford.edu}
}

\maketitle
%\thispagestyle{empty}



%-------------------------------------------------------------------------
%%%%%%%%% ABSTRACT
%-------------------------------------------------------------------------
\begin{abstract} % 1-2 paragraphs
  
  Begin text...

\end{abstract}



%-------------------------------------------------------------------------
%%%%%%%%% BODY TEXT
%-------------------------------------------------------------------------
\section{Introduction} % 0.5-1 page
% - Explain the problem and why it is important.
% - Give some background, if necessary.
% - Clearly state what the inputs and outputs are, be explicit.
%   - e.g. The input to our algorithm is a {YouTube video,3D image,etc}
%   - e.g. We then use a {CNN,LSTM,GAN,etc} to output a predicted {class,etc}

Begin text...



%-------------------------------------------------------------------------
\section{Related Work} % 0.5-1 page
% - Discuss at least 15 references of related work.
%   - Categorize existing papers based on approaches.
%   - e.g. Strengths, weaknesses.
%   - e.g. Which approaches were clever/good?
%   - e.g. What is state of the art?
%   - e.g. Do most people perform the task by hand?

Begin text...



%-------------------------------------------------------------------------
\section{Methods} % 2-3 pages
% - Describe learning algorithms, proposed algorithms, or theoretical proofs.
%   - Include relevant mathematical notations.
%   - e.g. State the GAN objectives, or include loss equation.
% - Describe  each algorithm used, and how it works.
%   - If using niche or cutting edge algorithms, explain over several paragraphs.

Begin text...



%-------------------------------------------------------------------------
\section{Dataset and Features} % 0.5-1 page
\subsection{Dataset}
% - Give details about your dataset.
%   - How many train/valid/test examples?
%   - What is the class label distribution of the dataset?
%   - Is there any preprocessing you did?
%   - What about normalization or data augmentation?
%   - What is the resolution of your images?
%   - If using video, how is it discretized?
% - Include examples from dataset (e.g. an image) in the report.
% - Include citation to dataset.

Begin text...


\subsection{Features}
% - Discuss features, and feature extraction techniques used.

Begin text...



%-------------------------------------------------------------------------
\section{Experiments, Reslts, and Discussion} % 2-3 pages
\subsection{Experiments}
% - Give details about hyperparameters you chose, and how you chose them.
% - Describe cross-validation scheme, if used.
% - List and explain metrics, provide equations if necessary.

Begin text...


\subsection{Results}
% - Include a mixture of tables and plots, both quantitative and qualitative!
% - Explain possible overfitting to training data, and what you did to mitigate.
% - Make sure plots have legends, axis labels, readable font sizes, etc.

Begin text...



%-------------------------------------------------------------------------
\section{Conclusion and Future Work} % 1-3 paragraphs
\subsection{Conclusion}
%  - Summarize report, reiterate key points.
%    - e.g. Which algorithms were the highest-performing?
%    - e.g. Why do you think that some algorithms worked better than others?

Begin text...


\subsection{Future Work}
%  - Discuss impact and future directions.
%    - e.g. Given more time, people, compute, etc., what would you explore?

Begin text...




%------------------------------------------------------------------------%
%%%%%%%%%----ALL SECTIONS ABOVE THIS POINT MUST FIT ON 8 PAGES---%%%%%%%%%
%------------------------------------------------------------------------%


%-------------------------------------------------------------------------
%%%%%%%%% REFERENCES
%-------------------------------------------------------------------------
{\small
\bibliographystyle{latex/ieee}
\bibliography{references}
}



%-------------------------------------------------------------------------
%%%%%%%%% APPENDICES
%-------------------------------------------------------------------------
\section{Appendices} % optional
% - Proof derivations
% - Equations, i.e. to minimize disruption of main paper flow









%------------------------------------------------------------------------%
%%%%%%%%%--------------------END DOCUMENT------------------------%%%%%%%%%
%------------------------------------------------------------------------%
\end{document}
